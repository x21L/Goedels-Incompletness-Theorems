\documentclass[aspectratio=169]{beamer}
% language and font set up
\usepackage[utf8]{inputenc}
\usepackage{amsmath}
\usepackage{amsfonts}
% additonal packages
\usepackage{graphicx}
% design
\usetheme{Berlin}
% \usecolortheme{beaver}
% informations
\title{Gödel's Incompletness Theorems}
\author{Lukas Wais}
\institute{Course 326.901}
\date{\today}

\begin{document}
% title frame
\frame{\titlepage}
% table of content
\begin{frame}
	\frametitle{Table of Contents}
	\tableofcontents
\end{frame}
% introduction
\section{Introduction}
\subsection{Who was Kurt Gödel}
\begin{frame}
	\frametitle{Kurt Gödel}
\end{frame}
\begin{frame}
	\frametitle{A quick reminder of Axioms}
	
	 \begin{Definition}[Axiom]
	 	Statements that are true without a formal proof of them. \\ For example:
	 	\[x = y \land y = z \implies x = z\]
	  	\begin{center}
	  		"It is possible to draw a straight line from any point to any other point"
	  	\end{center}
	 \end{Definition}
	 \begin{itemize}
	 	\item Any mathematical system starts out with a set of axioms
	 \end{itemize}
\end{frame}

\subsection{What is Completness?}
\begin{frame}
	\frametitle{Completness}
	 \begin{Definition}[Complete]
	 	A set of axioms is (syntactically, or negation-) complete if, for any statement in the axioms' language, that statement or its negation is provable from the axioms. \cite{smith}
	 \end{Definition}
\end{frame}
\section{Incompletness and Programming}
% citations
\begin{frame}{References}
    \bibliographystyle{ieeetr}
    \bibliography{bibfile}
\end{frame}
\end{document}
